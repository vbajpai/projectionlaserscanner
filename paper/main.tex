\documentclass[english]{llncs}

\usepackage[T1]{fontenc}
\usepackage[latin9]{inputenc}
\usepackage{babel}

%----------------------------------------------------------------------
%---------------------packages added on anuj' base -------------------
%----------------------------------------------------------------------
\usepackage[nolist]{acronym}
\usepackage{amsmath}
\usepackage{amsfonts}
\usepackage{amssymb}
%----------------------------------------------------------------------


%----------------------------------------------------------------------
%%----------------------- temporary packages --------------------------
%----------------------------------------------------------------------
\usepackage{scrtime}
\usepackage{prelim2e}
\usepackage{todonotes}
\renewcommand{\PrelimWords}{\relax}
\renewcommand{\PrelimText}{\footnotesize[\,\today\ at \thistime\,]}
%----------------------------------------------------------------------

\makeatletter
\newenvironment{keywords}{
       \list{}{\advance\topsep by0.35cm\relax\small
       \leftmargin=1cm
       \labelwidth=0.35cm
       \listparindent=0.35cm
       \itemindent\listparindent
       \rightmargin\leftmargin}
			 \item[\hskip\labelsep\bfseries Keywords:]}
     {\endlist}
\makeatother

\begin{document}

\frontmatter 
	\pagestyle{headings} 
\mainmatter 

\title{A Portable Low-Cost Archaeological 3D Imaging and Cataloging System}

\author{Vaibhav Bajpai\and Anuj Sehgal\and Daniel Cernea}

\institute{Computer Science, Jacobs University Bremen\\
					 Campus Ring 1, 28759 Bremen, Germany\\
\email{\{v.bajpai, s.anuj, d.cernea\}@jacobs-university.de}}

\titlerunning{A Portable Low-Cost Archaeological 3D 
						  Imaging and Cataloging System}

\authorrunning{Vaibhav Bajpai and Anuj Sehgal and Daniel Cernea}
\maketitle

\begin{abstract}

This paper introduces a cross-platform low-cost system for 3D object reconstruction using a projection-based laser scanner. It uses contact-free measurement techniques for 3D object reconstruction and fast surface registration using \ac{ICP}. The only hardware requirements are a simple commercial hand-held laser and a standard camera. The camera is initially calibrated using Tsai's camera calibration method so that its external and internal parameters are exactly known. The visible intersection with the background is used to find the exact 3D pose of the laser plane. This laser plane is used to triangulate new 3D point coordinates of the object’s surface. The point clouds obtained are processed using \ac{3DTK} which includes an automatic high-accurate registration process and a fast 3D viewer.

\end{abstract}

\begin{keywords}
	Camera Calibration, Hough Transformation, 
	3D Reconstruction, Scan Registration, 
	Hand-Held Laser, Image Filters, Color Models
\end{keywords}

\section{Introduction}
\label{section:introduction}
\begin{figure}[bht]
\centering
\includegraphics[width=0.5\linewidth]{figures/introduction}
\label{figure:introduction}
\caption{...}
\end{figure}

\todo{introduction}

\section{Related Work}
\label{section:relatedwork}
\todo{related work}
\todo{cite: David Laser Scanner}

\section{Approach}
\label{section:approach}
The complete pipeline starting from capturing the video of the laser sweeping
across an object to the end result of visualizing the 3D model of
reconstructed point cloud is primarily divided into five major steps as shown
in figure \ref{figure:pipeline}.

\begin{figure}[ht!]
\centering
\includegraphics[width=0.9\linewidth]{figures/pipeline}
\caption{Software Pipeline}
\label{figure:pipeline}
\end{figure}

\subsection{Data Acquisition}
\label{subsection:data-acquistion}
\begin{figure}[ht!]
\centering
\includegraphics[width=0.5\linewidth]{figures/introduction}
\caption{Data Acquisition}
\label{figure:acquisition}
\end{figure}

We used a standard digital camera to capture multiple runs of a hand-held laser sweeping across the object as shown in figure \ref{figure:acquisition}. Since the videos were stored in raw format which cannot be directly processed by OpenCV \cite{bradski:2008}, we used \texttt{mplayer} to extract individual frames at the rate of 5 frames per second.

\begin{verbatim}
$ mplayer -demuxer rawvideo \
          -rawvideo fps=5:w=1600:h=1200:yuy2 \
          -vo pnm:ppm $FILE	
\end{verbatim}

These frames were then later read into memory by calling the OpenCV routine \texttt{cvLoadImage()} with a \texttt{CV\_LOAD\_IMAGE\_UNCHANGED} flag. The routine allocates an image data structure and returns a pointer to a struct of type \texttt{IplImage}

\subsection{Camera Calibration}
\label{subsection:camera-calibration}
The first step in the process of reconstructing the 3D geometry of the object
is to establish a mathematical relationship between the natural units of the
camera with the physical units of the 3D world. We used camera calibration to
learn the internal parameters of the camera and its distortion coefficients.
The geometry is described in terms of camera's optical center and focal
length.

\begin{figure}[ht!]
\centering
\subfigure{\includegraphics[width=.40\linewidth]{figures/calibrate-1}}\quad
\subfigure{\includegraphics[width=.40\linewidth]{figures/calibrate-2}} \\
\subfigure{\includegraphics[width=.40\linewidth]{figures/calibrate-3}}\quad
\subfigure{\includegraphics[width=.40\linewidth]{figures/calibrate-4}}
\caption{Calculating the Camera's Intrinsic Parameters}
\label{figure:camera-calibration-intrinsics}
\end{figure}

We used OpenCV routines that are based on \cite{zhang:2000} \cite{brown:1971}
and used a planar chessboard pattern as our calibration object. We rotated and
translated the pattern to provide multiple views to get the precise
information about the intrinsic parameters of the camera as shown in figure
\ref{figure:camera-calibration-intrinsics}.  The OpenCV routine
\texttt{cvFindChessboardCorners()} is used to locate the corners and once we
had enough corners from multiple view images, we used
\texttt{cvCalibrateCamera2()} to get the intrinsic matrix $A$ as shown in
equation \ref{equation:calibrate}.


\begin{align}
	\label{equation:calibrate}
	s \times
	\begin{bmatrix}
		u \\ v \\	1 \\
	\end{bmatrix} &= A \cdot \begin{bmatrix}
															R \mid T
	 				  								\end{bmatrix}
										 \cdot \begin{bmatrix}
															x_w \\ y_w \\ z_w \\ 1
														\end{bmatrix} \\
	\text{where}~
	A &= \begin{bmatrix}
					f_x & 0 & c_x \\
					0 & f_y & c_y \\
					0 & 0 & 1 \\
 		 	 \end{bmatrix} \notag
\end{align}

The intrinsic matrix $A$ was later used to describe the pose
\footnote{combination of position and orientation} of the objects being
scanned by the laser relative to the coordinate system of the camera. In order
to determine this pose on both sides of the target object, the patterns were
masked out to allow individual calculation as shown in figure
\ref{figure:camera-calibration-extrinsics}. The parameters represented by
$\begin{bmatrix}R \mid T\end{bmatrix}$ could then be separately calculated for
both the sides by calling the OpenCV routine
\texttt{cvFindExtrinsicCameraParams2()}.

\begin{figure}[ht!]
\centering
\subfigure[$R_1 \mid T_1$]
{\includegraphics[width=.45\linewidth]{figures/calibrate-5}} \quad
\subfigure[$R_2 \mid T_2$]
{\includegraphics[width=.45\linewidth]{figures/calibrate-6}}
\caption{Calculating the Camera's Extrinsic Parameters}
\label{figure:camera-calibration-extrinsics}
\end{figure}


\subsection{Identification of 2D Laser Lines and Object Points}
\label{subsection:search-laser}
We used OpenCV routine \texttt{cvAbsDiff()} to calculate the single-channel image difference of the laser image from the reference image using equation \ref{equation:difference-image}. The resulted image difference is shown in figure \ref{figure:difference-image}

\begin{align}
	\label{equation:difference-image}				
	C = A &- B \\
	\text{where}~ 
	&A~ \text{is the laser image in figure \ref{subfigure:A} and} \notag \\
	&B~ \text{is the reference image in figure \ref{subfigure:A}} \notag
\end{align}


\begin{figure}[h!]
\centering
\subfigure[$A$]
{\includegraphics[width=.31\linewidth]{figures/difference-1}
 \label{subfigure:A}} \hfill
\subfigure[$B$]
{\includegraphics[width=.31\linewidth]{figures/difference-2}
\label{subfigure:B}} \hfill
\subfigure[$C$]
{\includegraphics[width=.31\linewidth]{figures/difference-3}
\label{subfigure:C}} \hfill
\label{figure:difference-image}
\caption{Using Image Difference to Find the Laser}
\end{figure}



\begin{figure}[h!]
\centering
\subfigure[Difference Image]
{\includegraphics[width=.45\linewidth]{figures/gauss-1}
\label{subfigure:A}} \quad
\subfigure[Gaussian Smoothened Image]
{\includegraphics[width=.45\linewidth]{figures/gauss-2}
\label{subfigure:B}} \hfill
\caption{Gaussian Smoothing the Difference Image}
\end{figure}


\begin{figure}[h!]
\centering
\subfigure[Image with Outliers]
{\includegraphics[width=.45\linewidth]{figures/colorthres-1}
\label{subfigure:A}} \quad
\subfigure[Image without Outliers]
{\includegraphics[width=.45\linewidth]{figures/colorthres-2}
\label{subfigure:B}} \hfill
\caption{Color Thresholding to Remove Outliers}
\end{figure}

\begin{figure}[h!]
\centering
\subfigure[Before Transformation]
{\includegraphics[width=.45\linewidth]{figures/hough-1}
\label{subfigure:A}} \quad
\subfigure[After Transformation]
{\includegraphics[width=.45\linewidth]{figures/hough-2}
\label{subfigure:B}} \hfill
\caption{Hough Transformation}
\end{figure}




\begin{itemize}
	\item Take a Difference Image to find the Laser
  \item Smoothen the Difference Image
  \item Color Threshold to remove everything else
  \item Apply Hough Transform
  \item Wrap the Points and Return the results
\end{itemize}

\subsection{Generation of Point Cloud}
\label{subsection:generate-pointcloud}
\begin{figure}[ht!]
\centering
\includegraphics[width=0.9\linewidth]{figures/pointcloudpaper}
\caption{Laser Triangulation \cite{winkelbach:2006}}
\label{figure:triangulation}
\end{figure}

The first step is to use the calculated camera extrinsics ($R \mid T$) for
each side of the target object along with the intrinsic parameters ($A$) to
transform each laser pixel ($P_c$) into 3D laser surface points ($P_w$) using
equation \ref{equation:3dlaserpoint}. The laser pixel points are expressed in
the homogenous coordinate system.

\begin{align}
	\label{equation:3dlaserpoint}
	P_w &= 	\underbrace{s \cdot R^{-1}
 					 							\cdot A^{-1}
 												\cdot P_c
										 }_\text{$\overrightarrow{b}$}
					-
					\underbrace{
											R^{-1} \cdot T
										 }_\text{$\overrightarrow{a}$} \\
	\text{where}~
	P_c &= \begin{pmatrix}
						u \\
						v \\
						1 \\
				 \end{pmatrix},
	\overrightarrow{a} = \begin{pmatrix}
													a_x \\
													a_y \\
													a_z \\
												\end{pmatrix},
	\overrightarrow{b} = \begin{pmatrix}
													b_x \\
													b_y \\
													b_z \\
												\end{pmatrix} \notag \\
  \text{and}~ s &= \frac{a_z}{b_z} \notag
\end{align}

Next in order to bring all the laser surface points into a common coordinate
system, we transformed all the 3D laser points from the right side of the
target object to the coordinate system of the left side using equation
\ref{equation:right2left}

\begin{align}
	\label{equation:right2left}
	P_{l} &= R_1^{-1} \times P_{r} - R_1^{-1}T_1 \\
	\text{where}~
	P_{r} &= \begin{bmatrix}
									R_2 \mid T_2
 				  \end{bmatrix} \times P_w \notag
\end{align}

With all the 3D laser surface points in a common coordinate system, we
randomly choose 3 points to generate the laser plane equation. Using the
coefficients of this equation, we defined the normal to the plane
$(\overrightarrow{N})$ using equation \ref{equation:laserplaneequation}.

\begin{align}
	\label{equation:laserplaneequation}
	A_x + B_y + C_z + D &= 0 \\
	\text{where}~
	 \overrightarrow{N} &=
	 \begin{pmatrix}
	  A \\
	  B \\
	  C \\
	 \end{pmatrix} \notag
\end{align}

The last step is to use the target object pixels $(P_c)$ and intersect them
with the laser plane equation represented by the normal $(\overrightarrow{N})$
to obtain the 3D surface points of the target object $(P_w)$ as shown in
figure \ref{figure:triangulation} using equation \ref{equation:3dobjectpoint}

\begin{align}
	\label{equation:3dobjectpoint}
	P_w &= s \times R^{-1}
 					 \times A^{-1}
					 \times P_c
					- R^{-1} \times T \\
	\text{where}~
	s &= \frac{\overrightarrow{N}\times\overrightarrow{a} - D}
						{\overrightarrow{b}\times\overrightarrow{N}}
  ~\text{and}~ P_c = \begin{pmatrix}
												u \\
												v \\
												1 \\
										 \end{pmatrix} \notag
\end{align}

In addition, we used OpenCV routine \texttt{cvGet2D()} to retrieve the RGB
color information for each object pixel and mapped it to the calculated
corresponding 3D object surface point. We used the reference image as the
source for retrieving the original color information since that image does not
have a laser line sweeping the target object.


\subsection{Point Cloud Processing and Registration}
\label{subsection:registration}
\begin{itemize}
	\item 3DTK
\end{itemize}


\section{Some Experimental Results}
\label{section:results}
\begin{figure}[ht!]
\centering
\subfigure{\includegraphics[width=.25\linewidth]{figures/results-1}}\quad
\subfigure{\includegraphics[width=.25\linewidth]{figures/results-2}}
\label{figure:results}
\caption{Results}
\end{figure}

The imaging system in addition to the source directory, expects a reference image without the laser stripe and two images of the individual background patterns used for calibration as input. The program saves the point cloud thus obtained in a destination directory which is used by the \ac{3DTK} components for processing and visualization as shown below. 

\begin{verbatim}
$ bin/projectionlaserscanner $SOURCE_DIR \
                             $REFERENCE_IMG \
                             $LEFT_CHECKERBOARD_PATTERN \
                             $RIGHT_CHECKERBOARD_PATTERN \
                             $DESTINATION_DIR \
$ bin/slam6D $DESTINATION_DIR
$ bin/show $DESTINATION_DIR
\end{verbatim}

The results thus obtained from the \texttt{show} program are shown in figure \ref{figure:results}.

\section{Evaluation and Comparison (maybe?)}
\label{section:evaluation-comparison}
\input{sections/evaluation-comparison}

\section{Conclusion}
\label{section:conclusion}
The process of data acquisition, point cloud generation and scan registrations
are currently performed offline. A foreseeable future work item is to make it
real time to allow the results to be viewed as they are being processed. A
performance evaluation with a larger dataset and a study comparing the results
with the David Laser Scanner also needs to be done.

We presented our implementation of an archaeological system for 3D object
reconstruction using a hand-held laser line projector and a web camera.  This
work provides a free and cross-platform alternative to the David Laser
Scanner. It uses contact-free triangulation with self-calibration of the laser
plane to generate the 3D object surface points. The point clouds obtained from
each scan are registered using \ac{SLAM} from \ac{3DTK} and viewed using its
fast viewer.


\bibliographystyle{splncs}
\bibliography{bibliography}

\begin{acronym}[ICP]
    \acro{ICP}{Iterative Closest Path}
		\acro{3DTK}{3D Toolkit}
\end{acronym}

\end{document}