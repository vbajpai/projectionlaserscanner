\begin{figure}[h!]
\centering
\subfigure[$X$]
{\includegraphics[width=.48\linewidth]{figures/difference-1}
 \label{subfigure:diff-A}} \hfill
\subfigure[$Y$]
{\includegraphics[width=.48\linewidth]{figures/difference-2}
\label{subfigure:diff-B}} \\
\subfigure[$Z$]
{\includegraphics[width=.48\linewidth]{figures/colorthres}
\label{subfigure:diff-C}} \hfill
\caption{Using Image Difference to Find the Laser}
\label{figure:difference-image}
\end{figure}

We use OpenCV routine \texttt{cvAbsDiff()} to calculate the image difference
of the laser image from the reference image using Eq.
\ref{equation:difference-image}. The resulted image difference is shown in
Fig. \ref{figure:difference-image}

\begin{align}
\label{equation:difference-image}
Z = X &- Y \\
\text{where}~
&X~ \text{is the laser image in figure \ref{subfigure:diff-A} and} \notag \\
&Y~ \text{is the reference image in figure \ref{subfigure:diff-B} and} \notag \\
&Z~ \text{is the difference image in figure \ref{subfigure:diff-C}} \notag
\end{align}

To reduce the noise in the difference image, we use the OpenCV routine
\texttt{cvSmooth()} to convolve the image with a Gaussian kernel function. It
not only helps to remove the camera artifacts but also reduces the information
content in the image. In order to remove all outliers and keep just the pixels
representing the red laser line, we use a pre-defined threshold value for the
intensity of the red pixels. In order to restrict this thresholding only along
the red channel, we use \texttt{cvSplit()} to split the three-channel (R,G,B)
difference image into separate one-channel image planes. We use
\texttt{cvGet2D()} and \texttt{cvSet2D()} to work on the scalar values of the
pixels.

%\begin{figure}[ht!]
%\centering
%\subfigure[Image with Outliers]
%{\includegraphics[width=.45\linewidth]{figures/gauss}
%\label{subfigure:colorthres-A}} \quad
%\subfigure[Image without Outliers]
%{\includegraphics[width=.45\linewidth]{figures/colorthres}
%\label{subfigure:colorthres-B}} \hfill
%\caption{Color Thresholding to Remove Outliers}
%\label{figure:color-thres}
%\end{figure}

%\begin{figure}[ht!]
%\centering
%\subfigure[Image with Outliers]
%{\includegraphics[width=.45\linewidth]{figures/gauss}
%\label{subfigure:colorthres-A}} \quad
%\subfigure[Image without Outliers]
%{\includegraphics[width=.45\linewidth]{figures/colorthres}
%\label{subfigure:colorthres-B}} \hfill
%\caption{Color Thresholding to Remove Outliers}
%\label{figure:color-thres}
%\end{figure}

We use the \ac{PPHT} \cite{kiryati:1991}, \cite{matas:2000} using the OpenCV
routine \texttt{cvHoughLines2()} to detect the laser lines on both sides of
the target object. The line end points of each line thus obtained are used to
draw the line using \texttt{cvLine()} as shown in Fig.
\ref{figure:hough-transform}.  The difference image is initially passed
through an edge detection phase, since the hough transform not only expects a
gray-scale image as input but the input is also treated as binary information
where the non-zero points are edge points of the image. Therefore, we use the
OpenCV routine \texttt{cvCvtColor} to convert the RGB difference image to gray
scale and \texttt{cvCanny()} to perform the Canny Edge Detection
\cite{canny:1986} before \ac{PPHT}. The two hough line equations on either
side of the object are used to find the laser line points, while the points
not identified as part of the hough line are taken as target object points.

\begin{figure}[ht!]
\centering
\subfigure[Before Transformation]
{\includegraphics[width=.46\linewidth]{figures/colorthres}
\label{subfigure:hough-A}} \quad
\subfigure[After Transformation]
{\includegraphics[width=.46\linewidth]{figures/hough}
\label{subfigure:hough-B}} \\
\caption{Hough Transformation}
\label{figure:hough-transform}
\end{figure}
