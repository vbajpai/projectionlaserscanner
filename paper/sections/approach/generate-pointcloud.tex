\begin{figure}[ht!]
\centering
\includegraphics[width=0.9\linewidth]{figures/pointcloudpaper}
\caption{Laser Triangulation \cite{winkelbach:2006}}
\label{figure:triangulation}
\end{figure}

The first step is to use the calculated camera extrinsics ($R \mid T$) for
each side of the target object along with the intrinsic parameters ($A$) to
transform each laser pixel ($P_c$) into 3D laser surface points ($P_w$) using
equation \ref{equation:3dlaserpoint}. The laser pixel points are expressed in
the homogenous coordinate system.

\begin{align}
	\label{equation:3dlaserpoint}
	P_w &= 	\underbrace{s \cdot R^{-1}
 					 							\cdot A^{-1}
 												\cdot P_c
										 }_\text{$\overrightarrow{b}$}
					-
					\underbrace{
											R^{-1} \cdot T
										 }_\text{$\overrightarrow{a}$} \\
	\text{where}~
	P_c &= \begin{pmatrix}
						u \\
						v \\
						1 \\
				 \end{pmatrix},
	\overrightarrow{a} = \begin{pmatrix}
													a_x \\
													a_y \\
													a_z \\
												\end{pmatrix},
	\overrightarrow{b} = \begin{pmatrix}
													b_x \\
													b_y \\
													b_z \\
												\end{pmatrix} \notag \\
  \text{and}~ s &= \frac{a_z}{b_z} \notag
\end{align}

Next in order to bring all the laser surface points into a common coordinate
system, we transformed all the 3D laser points from the right side of the
target object to the coordinate system of the left side using equation
\ref{equation:right2left}

\begin{align}
	\label{equation:right2left}
	P_{l} &= R_1^{-1} \times P_{r} - R_1^{-1}T_1 \\
	\text{where}~
	P_{r} &= \begin{bmatrix}
									R_2 \mid T_2
 				  \end{bmatrix} \times P_w \notag
\end{align}

With all the 3D laser surface points in a common coordinate system, we
randomly choose 3 points to generate the laser plane equation. Using the
coefficients of this equation, we defined the normal to the plane
$(\overrightarrow{N})$ using equation \ref{equation:laserplaneequation}.

\begin{align}
	\label{equation:laserplaneequation}
	A_x + B_y + C_z + D &= 0 \\
	\text{where}~
	 \overrightarrow{N} &=
	 \begin{pmatrix}
	  A \\
	  B \\
	  C \\
	 \end{pmatrix} \notag
\end{align}

The last step is to use the target object pixels $(P_c)$ and intersect them
with the laser plane equation represented by the normal $(\overrightarrow{N})$
to obtain the 3D surface points of the target object $(P_w)$ as shown in
figure \ref{figure:triangulation} using equation \ref{equation:3dobjectpoint}

\begin{align}
	\label{equation:3dobjectpoint}
	P_w &= s \times R^{-1}
 					 \times A^{-1}
					 \times P_c
					- R^{-1} \times T \\
	\text{where}~
	s &= \frac{\overrightarrow{N}\times\overrightarrow{a} - D}
						{\overrightarrow{b}\times\overrightarrow{N}}
  ~\text{and}~ P_c = \begin{pmatrix}
												u \\
												v \\
												1 \\
										 \end{pmatrix} \notag
\end{align}

In addition, we used OpenCV routine \texttt{cvGet2D()} to retrieve the RGB
color information for each object pixel and mapped it to the calculated
corresponding 3D object surface point. We used the reference image as the
source for retrieving the original color information since that image does not
have a laser line sweeping the target object.
