The first step in the process of reconstructing the 3D geometry of the object
is to establish a mathematical relationship between the natural units of the
camera with the physical units of the 3D world. We used camera calibration to
learn the internal parameters of the camera and its distortion coefficients.
The geometry is described in terms of camera's optical center and focal
length.

\begin{figure}[ht!]
\centering
\subfigure{\includegraphics[width=.40\linewidth]{figures/calibrate-1}}\quad
\subfigure{\includegraphics[width=.40\linewidth]{figures/calibrate-2}} \\
\subfigure{\includegraphics[width=.40\linewidth]{figures/calibrate-3}}\quad
\subfigure{\includegraphics[width=.40\linewidth]{figures/calibrate-4}}
\caption{Calculating the Camera's Intrinsic Parameters}
\label{figure:camera-calibration-intrinsics}
\end{figure}

We used OpenCV routines that are based on \cite{zhang:2000} \cite{brown:1971}
and used a planar chessboard pattern as our calibration object. We rotated and
translated the pattern to provide multiple views to get the precise
information about the intrinsic parameters of the camera as shown in figure
\ref{figure:camera-calibration-intrinsics}.  The OpenCV routine
\texttt{cvFindChessboardCorners()} is used to locate the corners and once we
had enough corners from multiple view images, we used
\texttt{cvCalibrateCamera2()} to get the intrinsic matrix $A$ as shown in
equation \ref{equation:calibrate}.


\begin{align}
	\label{equation:calibrate}
	s \times
	\begin{bmatrix}
		u \\ v \\	1 \\
	\end{bmatrix} &= A \cdot \begin{bmatrix}
															R \mid T
	 				  								\end{bmatrix}
										 \cdot \begin{bmatrix}
															x_w \\ y_w \\ z_w \\ 1
														\end{bmatrix} \\
	\text{where}~
	A &= \begin{bmatrix}
					f_x & 0 & c_x \\
					0 & f_y & c_y \\
					0 & 0 & 1 \\
 		 	 \end{bmatrix} \notag
\end{align}

The intrinsic matrix $A$ was later used to describe the pose
\footnote{combination of position and orientation} of the objects being
scanned by the laser relative to the coordinate system of the camera. In order
to determine this pose on both sides of the target object, the patterns were
masked out to allow individual calculation as shown in figure
\ref{figure:camera-calibration-extrinsics}. The parameters represented by
$\begin{bmatrix}R \mid T\end{bmatrix}$ could then be separately calculated for
both the sides by calling the OpenCV routine
\texttt{cvFindExtrinsicCameraParams2()}.

\begin{figure}[ht!]
\centering
\subfigure[$R_1 \mid T_1$]
{\includegraphics[width=.45\linewidth]{figures/calibrate-5}} \quad
\subfigure[$R_2 \mid T_2$]
{\includegraphics[width=.45\linewidth]{figures/calibrate-6}}
\caption{Calculating the Camera's Extrinsic Parameters}
\label{figure:camera-calibration-extrinsics}
\end{figure}
