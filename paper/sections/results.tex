\begin{figure}[ht!]
\centering
\subfigure{\includegraphics[width=.30\linewidth]{figures/results/100deg}}\quad
\subfigure{\includegraphics[width=.30\linewidth]{figures/results/80deg}}\quad
\subfigure{\includegraphics[width=.30\linewidth]{figures/results/60deg}}\\
\subfigure{\includegraphics[width=.30\linewidth]{figures/results/40deg}}\quad
\subfigure{\includegraphics[width=.30\linewidth]{figures/results/20deg}}\quad
\subfigure{\includegraphics[width=.30\linewidth]{figures/results/0deg}}\\
\caption{Results}
\label{figure:results}
\end{figure}

The imaging system in addition to the source directory, expects a reference
image without the laser stripe and two images of the individual background
patterns used for calibration as input. The program saves the point cloud thus
obtained in a destination directory which is used by the \ac{3DTK} components
for processing and visualization as shown below.

\begin{verbatim}
$ bin/projectionlaserscanner $SOURCE_DIR \
                             $REFERENCE_IMG \
                             $LEFT_CHECKERBOARD_PATTERN \
                             $RIGHT_CHECKERBOARD_PATTERN \
                             $DESTINATION_DIR \
$ bin/slam6D $DESTINATION_DIR
$ bin/show $DESTINATION_DIR
\end{verbatim}

The results thus obtained from the \texttt{show} program are shown in figure
\ref{figure:results}.
